%!TeX root=../tese.tex

\chapter{Background}
\label{cap:background}

In this chapter, we present the theoretical background of description logics, 

\improvement[inline]{Add initial description of the chapter}

\section{Description logics}
Description logics (DLs) are used to represent knowledge, such as the semantic of words,  people and their relations, and medical terms. These scenarios require precise specification and meaning so that different systems behave the same way. The first DL modeling languages appeared in the mid-1980s and have an important role in the context of the Semantic Web, an initiative to represent web content in a form that is more machine friendly \citep{krotzsch2012description}.

As their name suggests, DLs are logics; indeed, most of them are fragments of first-order logic. This relation with logics is what provides their precise specification, called \emph{formal semantics}. Also, it equips their languages with a formal deduction to \emph{infer} additional information, and the computation of these inferences is called \emph{reasoning}. The performance of algorithms for reasoning strongly relies on the expressiveness of the DL: fast algorithms usually exist for lightweight logics. Then, there is not just a single DL because the balance between expressiveness and performance depends on the application. \citep{krotzsch2012description}

\subsection{Building blocks of description logic ontologies}
A DL is composed of concepts, roles, and individual names. Concepts are sets of individuals, roles are binary relations between individuals and individual names are single individuals in the domain.

For example, an ontology\todo{Maybe explain what is an ontology} modeling the situation in \autoref{exmp:real-example} can use the concepts \textsf{Patient}, to represent the set of all patients in the hospital, and \textsf{Symptom}, to represent the set of all symptoms; roles such \textsf{hasSymptom}, to represent the binary relation between patients and symptoms; and individual names such as \textsf{mary} and \textsf{s1}, to represent the individuals Mary and Mary's symptoms.

Additionally, DLs allows us to describe more complex situations, creating new concepts and roles from the previously defined ones. 

Some concept constructors provide boolean operations similar to that found in set theory and logic expressions. For example, if we want to describe the set of individuals that are both fever and cough, we could use the \emph{conjunction} operator, as follows
\[
	\textsf{Fever} \sqcap \textsf{Cough}.
\]

We can link concepts and roles using role restrictions. For example, to describe all individuals that are suspect of some disease that is COVID-19, we use the \emph{existential restriction}
\[
	\exists \textsf{suspectOf}.\textsf{COVID-19}.
\]

Also, to define concepts with only one individual we use \emph{nominals} like $\{\textsf{mary}\}$.

More expressive logics can have other operations such as \emph{disjunction} ($C \sqcup D$), \emph{negation} ($\lnot C$), \emph{universal restriction} ($\forall r.C$) and \emph{number restrictions} ($\leq\!\! n\, r.C$).  

To capture knowledge about the world, DL ontologies also allow us to describe relations between concepts, roles, and individual names. For example, the fact that all fevers are symptoms is represented by the \emph{concept inclusion}
\[
	\textsf{Fever} \isa \textsf{Symptom};	
\]
the knowledge that someone that has a symptom which is caused by some disease is suspect of that disease can be expressed by the \emph{role inclusion} with a \emph{role composition}
\[
	\textsf{hasSymptom} \, \circ \, \textsf{hasCause} \isa \textsf{suspectOf};	
\]
and the fact that Mary is a patient of the hospital and has symptoms is represented by the \emph{assertions} $\textsf{Patient}(\textsf{mary})$ and $\textsf{hasSymptom(mary, s1)}$.

After that, if we have a set of these relations, one could ask if there is a set of individuals, or instances, that satisfies these relations, which is called an \emph{interpretation}. Interpretations can be understood as the assignment of meaning to logical terms in an ontology. Because a DL usually considers all the possibles situations, property that is sometimes referred to as \emph{open world assumption}, an ontology can have multiple satisfiable interpretations. The fewer restrictions it has, the more interpretations satisfy this ontology. The computational complexity to find the existence of these interpretations is one of the key aspects to choose different DL fragments.

These terms will be formally defined in the \autoref{sec:el}, in the case of the DL $\elpp$.

\subsection{Description logic fragments and OWL}

There are many DL fragments. Each subset of features, like those described previously, can lead to different fragments of first-order logic. For example, the logic $\mathcal{ALC}$ does not allow role inclusions and admits only $\sqcap, \sqcup, \lnot, \exists$ and $\forall$ as concept constructors; their best reasoning algorithms, however, are worst-case exponential time. On the other hand, the $\el$ logic allows only $\sqcap$ and $\exists$ as concept constructors, and its reasoning algorithms are polynomial time.

To express DL ontologies, the World Wide Web Consortium (W3C) designed the OWL~2 Web Ontology Language (OWL 2) \citep{owl2}. This declarative language is part of the W3C's Semantic Web technology stack and comes with various syntaxes, such as RDF/XML. Because of this use on the web, names in OWL are \emph{international resource identifiers} (IRIs).

\section{The description logic \texorpdfstring{$\elpp$}{𝓔𝓛++}}
\label{sec:el}

$\elpp$ is an extension of the DL $\el$ \citep{Baader2005a}. It was created with large bio-health ontologies in mind, such as SNOMED-CT, the NCI thesaurus, and Galen, and became an official OWL 2 profile \citep{owl2}. We concentrate on presenting $\elpp$ without concrete domains.

\subsection{Syntax}
In $\elpp$, \emph{concept descriptions} are defined inductively from a set $\Nc$ of \emph{concept names}, a set $\Nr$ of \emph{role names} and set $\Ni$ of \emph{individual names} as follows:
\begin{itemize}
	\item $\top$, $\bot$ and concept names in $\Nc$ are concept descriptions;
	\item if $C$ and $D$ are concept descriptions, $C \sqcap D$ is a concept description;
	\item if $C$ is a concept description and $r \in \Nr$, $\exists r.C$ is a concept description;
	\item if $a \in \Ni$, $\{a\}$ is a concept description.
\end{itemize}

To represent knowledge using concept descriptions, we need to define facts (axioms and role inclusions) and assertions.

An \emph{axiom}, or a \emph{general concept inclusion} (GCI), is an expression of the form $C \isa D$, where $C$ and $D$ are concept inclusions. Also, we write $C \equiv D$ to represent the axioms $C \isa D$ and $D \isa C$. A \emph{role inclusion} (RI) is an expression of the form $r_1 \circ \cdots \circ r_k \isa r$, where $r_1, \dots r_k, r \in \Nr$.  The symbol ``$\circ$'' denotes composition of binary relations. A \emph{constraint box} (CBox) is a finite set of GCIs and a finite set of RIs.

Similarly, a \emph{concept assertion} is an expression of the form $C(a)$ and a \emph{role assertion}, $r(a, b)$, where $C$ is a concept description, $a, b \in \Ni$ and $r \in \Nr$. A finite set of concept assertions and role assertions is an \emph{assertional box} (ABox).

Then, an $\elpp$ \emph{knowledge base} $\K$ (KB) is a pair ($\mathcal{C}, \mathcal{A}$), where $\mathcal{C}$ is a CBox and $\mathcal{A}$ is an~ABox.

\subsection{Semantics}
The semantics of $\elpp$ are given by \emph{interpretations} $\I = (\Delta^\I, \cdot^\I)$. The \emph{domain} $\Delta^\I$ is a non-empty set of individuals, and the \emph{interpretation function} $\cdot^\I$ maps each concept name $A \in \Nc$ to a subset $A^\I$ of $\Delta^\I$, each role name $r$ to a binary relation $r^\I \subseteq \Delta^\I \times \Delta^\I$, and each individual name $a$ to an element $a^\I \in \Delta^\I$. The extension of $\cdot^\I$ for an arbitrary concept description is inductively defined by the third column of \autoref{table:elpp-syntax-semant}.

\begin{table}
	\centering
	\begin{tabular}{@{}rcc@{}}
		\toprule
		Name                    & Syntax                              & Semantics                                                                                        \\
		\midrule
		top                     & $\top$                              & $\Delta^\I$                                                                                      \\
		bottom                  & $\bot$                              & $\emptyset$                                                                                      \\
		nominal                 & $\{a\}$                             & $\{a^\I\}$                                                                                       \\
		conjunction             & $C \sqcap D$                        & $C^\I \cap D^\I$                                                                                 \\
		existential restriction & $\exists r.C$                       & $\{ x \in \Delta^\I \, | \, \exists y \in \Delta^\I \, : \, (x, y) \in r^\I \land y \in C^\I \}$ \\
		GCI                     & $C \isa D$                          & $C^\I \subseteq D^\I$                                                                            \\
		RI                      & $r_1 \circ \cdots \circ r_k \isa r$ & $r_1^\I \circ \cdots \circ r_k^\I \subseteq r^\I$                                                \\
		concept assertion       & $C(a)$                              & $a^\I \in C^\I$                                                                                  \\
		role assertion          & $r(a, b)$                           & $(a^\I, b^\I) \in r^\I$                                                                          \\
		\bottomrule
	\end{tabular}
	\caption{Syntax and semantics of $\elpp$ without concrete domains}
	\label{table:elpp-syntax-semant}
\end{table}

The interpretation $\I$ \emph{satisfies}:
\begin{itemize}
	\item an axiom $C \isa D$ if $C^\I \subseteq D^\I$ (represented as $I \models C \isa D$);

	\item a RI $r_1 \circ \cdots \circ r_k \isa r$ if $r_1^\I \circ \cdots \circ r_k^\I \subseteq r^\I$ (represented as $I \models r_1 \circ \cdots \circ r_k \isa r$);

	\item an assertion $C(a)$ if $a^\I \in C^\I$ (represented as $I \models C(a)$);

	\item an assertion $r(a, b)$ if $(a^\I, b^\I) \in r^\I$ (represented as $I \models r(a, b)$).
\end{itemize}

Also, we say that $\I$ is a \emph{model} of:
\begin{itemize}
	\item a CBox $\cbox$ if it satisfies every axiom and RI in $\cbox$ (represented as $\I \models \cbox$);
	\item an ABox $\abox$ if it satisfies every assertion in $\abox$ (represented as $\I \models \abox$);
\end{itemize}

Then, an important problem in $\elpp$ is to determine its $consistency$, that is if $\abox$ and $\cbox$ have a common model, which is in PTime \citep{Baader2005a}.

\subsection{Normal form}
We can convert an $\elpp$ knowledge base into a normal form, in polynomial time, by introducing new concept and role names \citep{Baader2005a}.

First, there is no need of explicit ABox, because $\I \models C(a)\iff \I \models \{a\} \isa C$ and $\I \models r(a, b) \iff \{a\} \isa \exists r.\{b\}$. In other words, a knowledge base can be represented by just a CBox, by transforming assertions in axioms.

In addition, given a CBox $\cbox$, consider the set $\bc$ of \emph{basic concept descriptions}, which is the smallest set of concept descriptions that contains the top concept $\top$, all concept names used in $\cbox$ and all concepts of the form $\{a\}$ used in $\cbox$.

Then, every axiom can be represented in the following normal form, where $C_1, C_2 \in \bc$, $D \in \bc \cup \{\bot\}$:
\begin{align*}
	C_1            & \isa D             \tag{simple}           \\
	C_1            & \isa \exists r.C_2 \tag{existential-head} \\
	\exists r.C_1  & \isa D             \tag{existential-body} \\
	C_1 \sqcap C_2 & \isa D  \tag{conjunctive-body}
\end{align*}
And every RI are of the form $r \isa s$ or $r_1 \circ r_2 \isa s$.

\begin{example}
	\label{exmp:el-cbox-def}
	Consider the following CBox $\CC_{exa}$ representing the situation in \autoref{exmp:real-example}. On the left, we have basic knowledge of diseases and, on the right, the specific knowledge about Mary. Note that, for simplicity, it is not in normal form.

		{
			\sffamily
			\begin{center}
				\begin{minipage}{0,4\textwidth}
					\fontsize{10}{14}
					\selectfont
					Fever $\isa$ Symptom\\
					Cough $\isa$ Symptom\\
					COVID-19 $\isa$ Disease\\
					Symptom $\isa \, \exists$hasCause.Disease\\
					Patient $\isa\, \exists$hasSymptom.Symptom\\
					CovidPatient $\equiv\, \exists$suspectOf.COVID-19\\
					hasSymptom $\circ$ hasCause $\isa$ suspectOf\\
				\end{minipage}
				\hspace{10pt}
				\begin{minipage}{0,4\textwidth}
					\fontsize{10}{14}
					\selectfont
					$\{$mary$\}$ $\isa$ Patient\\
					$\{$s1$\}$ $\isa$ Fever $\sqcap$ Cough\\
					$\{$mary$\}$ $\isa \, \exists$hasSymptom.$\{$s1$\}$\\
				\end{minipage}
			\end{center}
		}

	Because CBoxes can only represent facts, there is no way to describe uncertain knowledge. Even though, in cases when which of them are true, we could define three axioms

	\begin{description}
		\item { \sffamily $Ax_1 := $ Fever $\isa \exists$hasCause.COVID-19}, when fever is actually caused by COVID-19;
		\item { \sffamily $Ax_2 :=$ Fever $\sqcap$ Cough $\isa \exists$hasCause.COVID-19 }, when both fever and cough are caused by COVID-19;
		\item { \sffamily $Ax_3 :=$ COVID-19 $\isa \bot$}, when there are no presence of COVID-19 in the hospital.
	\end{description}

	In the following sections, it will be presented how to add these axioms in a KB with probabilistic properties.
\end{example}

We want to model uncertain information using DLs. However, it has been proved that, by adding probabilistic reasoning capabilities to $\elpp$, the complexity reaches NP-completeness \citep{Fin2019b}. Then, it is necessary to reduce the expressiveness of this language.

\section{Graphic \texorpdfstring{$\el$}{𝓔𝓛} (\texorpdfstring{$\gel$}{𝓖𝓔𝓛})}
\label{sec:gel}

\emph{Graphic} $\el$ ($\gel$) is a fragment of $\elpp$ in which every axiom and RI are in normal form and no conjunctive-body axiom is allowed \citep{Fin2020}. The semantics are the same as that of $\elpp$.

\begin{example}
	\label{exmp:gel-def}
	Since there are conjunctive-body axioms in the CBox in \autoref{exmp:el-cbox-def}, we need to modify this knowledge in order to represent it in $\gel$. First, we substitute every concept description {\sffamily Fever $\sqcap$ Cough} by a new basic concept  {\sffamily FeverAndCough}. After that, we add axioms {\sffamily FeverAndCough $\isa$ Fever} and {\sffamily FeverAndCough $\isa$ Cough} to CBox.
\end{example}

The name of this fragment comes from the fact that each GCI can be represented as arrows in a graph where nodes are basic concepts. This representation is useful for the development of algorithms and it is used to define its SAT decision.

\subsection{Graphical representation}

Consider a $\gel$-CBox $\CC$ with $n_R$ roles, its graphical representation is a edge-labeled graph $G(\CC) := (N, E, \ell)$, where $N$ is a set of nodes, $E \subseteq N^2$ is a set of directed edges and $\ell: E \to \{0, 1, \dots, n_R\}$ \todo{How do I define $\ell$ with the addition of artificial edges of the form $u_{r.C}$?} is a labeling function.

In addition, \citet{Fin2020} defines the following notation. The set $E_i \subseteq E$ is the set of all edges $e$ such that $\ell(e) = i$. We write $X_1 \to_i X_2$ if there is an edge $(X_1, X_2) \in E_i$, and $X_1 \nto_i X_2$ if $(X_1, X_2) \not\in E_i$. The expression $X \to_i^* Y$ represents the reflexive transitive closure of $\to_i$, which is the existence of a path in the graph of size $\geq 0$, starting in $X$, ending in $Y$, and only going through edges in $E_i$, for $0 \leq i \leq n_R$. Finally, $X \rightsquigarrow Y$ represents a path from $X$ to $Y$ using any type of edge.     

Then, the graph $G(\cbox) = (N, E, \ell)$ can be constructed from a CBox $\cbox$ with the following steps:
\begin{enumerate}
	\item for each concept $\exists r.C$ in a existential-body axiom of the form $\exists r.C \isa D$ create a \emph{virtual concept} ``$\exists r.C$''. The set of all virtual concepts in $\CC$ is called $\textsf{VC}_\CC$;
	\item the set $N$ of nodes is obtained from the basic concepts of $\CC$, an initial-node symbol $Init$, the bottom concept $\bot$ and the set of virtual concepts of $\CC$ as follows:
	\[ N := \{Init, \bot\} \cup \textsf{BC}_\CC \cup \textsf{VC}_\CC; \]
	\item if $C \isa D \in \CC$ then $C \to_0 D$; \todo[inline]{In the last paragraph, I wrote ``We write $X_1 \to_i X_2$ if there is an edge $(X_1, X_2) \in E_i$''. So the right notation is $C \to_0 D$ instead of $C \to_0 D \in E_0$ (the one you wrote in the paper), isnt it?  Doesnt $X_1 \to_i X_2$ already mean the existence of the edge in $E$? }
	\item if $C \isa \exists r_i.D$ then $C \to_i D$;
	\item if $\exists r_i.C \isa D$ then $\vc{\exists r_i.C} \to_0 D$, $\vc{\exists r_i.C} \to_i C$ and $C \to_{u_{r_i.C}} \vc{\exists r_i.C}$; \todo{Need to review this. I dont think the theory is finalized}
	\item $Init \to_0 \top$; \todo{The original rule was ``if $\top$ occurs in $\CC$ then $Init \to_0 \top$''. Can I change this?}
	\item for every node of the form $\{a_i\} \in N$, $Init \to_0 \{a_i\}$.
\end{enumerate}

\begin{example}
	Consider the CBox in \autoref{exmp:gel-def} and the uncertain information in \autoref{exmp:el-cbox-def}. Its graphical representation is displayed in the \autoref{fig:gel-graph}. The $\to_0$-edges are represented by continuous black arrows, the $\to_i$-edges, $i \geq 1$, are represented by blue labeled arrows and red dotted arrows indicate that source is uncertain information.
	
	Note that the axiom \textsf{CovidPatient $\equiv\, \exists$suspectOf.COVID-19} implies the existential-body axiom \textsf{$\exists$suspectOf.COVID-19 $\isa$ CovidPatient}. Therefore, it generates: 
	\begin{itemize}
		\item the virtual concept ``\textsf{suspectOf.COVID-19}'';
		\item \textsf{``suspectOf.COVID-19''$\,\to\,$CovidPatient};
		\item \textsf{``suspectOf.COVID-19''$\,\to_{\textsf{suspectOf}}\,$CovidPatient};
		\item and \textsf{COVID-19$\,\to_{u_{\text{suspectOf.Covid}}}\,$ ``suspectOf.COVID-19''}. \todo[inline]{Is this the right notation? Doesn't the label need to be an integer?}
	\end{itemize}
\end{example}

\tikzstyle{block} = [draw, rectangle, rounded corners=.2cm, fill=black!5, draw=white]
\tikzstyle{dashblock} = [draw, rectangle, dashed, rounded corners=.2cm, fill=black!5, draw=black!30, line width=0.8pt]
\tikzstyle{arrow} = [>=latex, line width=0.8pt]

\begin{figure}[ht]
	\centering
	\small
	\begin{tikzpicture}[auto, node distance=1.5cm, ->]
		\tikzset{every node/.style={minimum height=0.75cm}}

		\sffamily
		\node (init) {Init};
		\node [below=of init] (s1) {$\{$s1$\}$};
		\node [below=of s1] (mary) {$\{$mary$\}$};

		\node [block, right=0.5cm of mary] (patient) {Patient};
		\node [block, right=of s1] (fevcough) {FeverAndCough};
		\node [block, below=0.5cm of fevcough] (cough) {Cough};
		\node [block, right=of fevcough] (fever) {Fever};
		\node [block, below=of fever] (symptom) {Symptom};
		\node [block, above=of fever] (covid) {COVID-19};
		\node [block, right=of fever] (disease) {Disease};
		\node [dashblock, above=of covid] (suspofcov) {``suspectOfCOVID-19''};
		\node [block, left=0.5cm of suspofcov] (covidpat) {covidPatient};
		\node [above=of fevcough, yshift=-0.15cm] (bot) {$\bot$};

		\path [arrow]
		(init) edge (s1)
		(init) edge[bend right = 70] (mary)
		(s1) edge (fevcough)
		(mary) edge (patient)
		(covid) edge[bend right = -45] (disease)
		(fevcough) edge (fever)
		(fevcough) edge (cough)
		(cough) edge (symptom)
		(fever) edge (symptom)
		(suspofcov) edge (covidpat)
		;

		\path [arrow, NavyBlue]
		(mary) edge node [midway, fill=white, anchor=center] {hasSymptom} (s1)
		(patient) edge node [midway, fill=white, anchor=center] {hasSymptom} (symptom)
		(covidpat) edge[bend right = 40] node [near start, fill=white, anchor=center] {suspectOf} (covid.170)
		(symptom) edge[bend right = 45] node [midway, fill=white, anchor=center] {hasCause} (disease)
		(suspofcov) edge[bend right = 60] node [midway, fill=white, anchor=center] {suspectOf} (covid)
	(covid) edge[bend right = 60] node [midway, fill=white, anchor=center] {$u_{\text{suspectOf.Covid}}$} (suspofcov)
		;

		\path [arrow, dotted, FireBrick]
		(fevcough) edge node [midway, fill=white, anchor=center] {hasCause} (covid)
		(fever) edge node [midway, fill=white, anchor=center] {hasCause} (covid)
		;

		\path [arrow, dotted, FireBrick]
		(covid.190) edge (bot)
		;

	\end{tikzpicture}
	\caption{Graphical representation of the ontology in \autoref{exmp:gel-def}.}

	\label{fig:gel-graph}
\end{figure}

\subsection{SAT decision}

\improvement[inline]{Need to describe the graph completion rules}

\todo[inline]{Question about this part: do we really need the graph completion to define the SAT decision? Isn't it true that a $\gel$ CBox $\CC$ is \textbf{unsatisfiable}  iff $Init \rightsquigarrow \bot$  (using any type of edge) in a graph $G(\CC)$ \textbf{without} completion? I couldn't find any counterexample...}
After the graph completion, the satisfiability (SAT) decision of a $\gel$ CBox can be reduced to a path search in a graph, that is, a $\gel$ CBox is unsatisfiable iff $Init \to_0^* \bot$. 

\section{MaxSAT for \texorpdfstring{$\gel$}{𝓖𝓔𝓛}}
\label{sec:maxgel}

Before focusing on the probabilistic extension for $\gel$, we need to define its MaxSAT problem, which will compose further the probabilistic reasoner.

The \emph{weighted partial maximum satisfiability problem for} $\gel$ ($\gel$-MaxSAT) can be defined as follows: given a potentially inconsistent weighted CBox $\cbox$, we want to find the maximal satisfiable subset of axioms; since it is partial, some axioms must be present in this subset. Usually, partiality can be modeled assigning infinite weights to the axioms that must not be excluded.

A \emph{weighted} CBox is a pair $\tuple{\cbox, w}$ where $\cbox$ is a CBox and $w: \cbox \to \Ratios \cup \{\infty\}$ is a weight function, which maps axioms in $\cbox$ to weights. The infinite weight is used to represent axioms that must not be excluded in the maximal satisfiable subset. Also, it is defined that RIs of the form $r \isa s$ and $r_1 \circ r_2 \isa s$ always have infinite weight.

Then, given a weighted CBox $\tuple{\cbox, w}$, a solution for the weighted partial $\gel$-MaxSAT problem is a set $\cbox_{max} \subseteq \cbox$ such that:
\begin{itemize}
	\item $\cbox_{max}$ is satisfiable; and
	\item $\cbox_{max} \models C \isa D$ if $w(C \isa D) = \infty$; and
	\item the sum of finite weights in $\cbox_{max}$ is maximal.
\end{itemize}

\section{Probabilistic \texorpdfstring{$\gel$}{𝓖𝓔𝓛}}
\label{sec:pgel}

Probability in $\gel$ is constructed from a \emph{probability function} $P$ \citep{Fin2020}. Consider a finite number of interpretation, $\I_1, \dots, \I_m$, we define the probability function $P: \{\I_1, \dots, \I_m\} \to \Ratios$, such that $P(\I_i) \geq 0$ and $\sum_{i = 1}^m P(\I_i) = 1$. We can also define the probability of an axiom $C \isa D$ as follows

\[
	P(C \isa D) = \sum_{\I_i \models C \isa D} P(\I_i).
\]

A \emph{probabilistic knowledge base} is a pair $\tuple{\CC, \PP}$, where $\CC$ is a CBox and $\PP$ is a PBox. A PBox is a set of $k$ linear constraints over $n$ axioms, of the form

\begin{equation}
	\label{eq:pbox-restric}
	\sum_{j = 1}^n a_{ij} \cdot P(C_j \isa D_j) \leq b_i; \quad 1 \leq i \leq k.
\end{equation}

We can define the \emph{satisfiability problem} for this probabilistic KB (PGEL-SAT) as deciding if it is consistent or not. If it is consistent, the solution is a set of interpretations $\{\I_1, \dots, \I_m \}$ and a probability function $P: \{\I_1, \dots, \I_m \} \to \Ratios^+$ such that $\sum_{i = 1}^m P(\I_i) = 1$, $P(C \isa D) = 1$ for $C \isa D \in \CC$ (axioms in CBox are certain) and P verifies all linear constraints in $\PP$.
% 
\begin{example}
	Now we can model the uncertain situation stated in \autoref{exmp:real-example} using the probability knowledge base $\tuple{\CC_{exa}, \PP_{exa}}$, where $\CC_{exa}$ is the CBox from \autoref{exmp:gel-def} and $\PP_{exa}$ is given by
	\begin{align*}
		\PP_{exa} := \{ \quad P(Ax_2) - P(Ax_1) & = 0.2,           \\
		P(Ax_3)                                 & = 0.9  \quad \}.
	\end{align*}

	Then, we need a polynomial algorithm to find if this probabilistic KB is consistent.
\end{example}
% 
\subsection{Linear algebraic view}

The PGEL-SAT problem was also defined by \citet{Fin2020} in a linear algebraic view, which is useful to develop its polynomial reasoning algorithm. It was shown that a probabilistic KB $\tuple{\CC, \PP}$ is satisfiable iff the linear equation $C \cdot x = d$ has a solution $x \geq 0$, where
% 
\begin{equation}
	\label{pbox-linear}
	C := \begin{bmatrix}
		-I_n           & M_{n \times m} \\
		A_{k \times n} & 0_{k \times m} \\
		0'_n           & 1'_{m}
	\end{bmatrix}
	%
	\quad
	x := \begin{bmatrix}
		p_n \\
		\pi_m
	\end{bmatrix}
	%
	\quad
	d := \begin{bmatrix}
		0_n \\
		b_k \\
		1
	\end{bmatrix}
\end{equation}

and
\begin{itemize}
	\item $A_{k \times n}$ is a $k \times n$ matrix whose elements $a_{ij}$ are given by \autoref{eq:pbox-restric};
	\item $b_k$ is a $k$ vector whose elements $b_i$ are also given by \autoref{eq:pbox-restric};
	\item $M_{n \times m}$ is a $n \times m$ matrix given by the following steps:

	      Consider an interpretation $\I$ model of $\CC$, its corresponding vector in $\PP$ is  a $\{0, 1\}$-vector $y$ such that $y_i = 1$ iff $\I \models C_i \isa D_i$, for $1 \leq i \leq n$.

	      Then, given a set of interpretations $\I_1, \dots, \I_m$, we define $M_{n \times m}$ a matrix whose column $M^j$ is $\I_j$'s corresponding vector in $\PP$;
	\item $I_n$ is the $n$-dimensional identity matrix;
	\item $0_n$ is a column $0$-vector of size $n$ (similarly for $1_n$);
	\item $0'_n$ is the previous vector's transpose;
	\item $0_{k \times m}$ is a $0$-matrix of shape $k \times m$;
	\item $p_n$ is a vector of size $n$ which corresponds to the probability of axioms occurring in \autoref{eq:pbox-restric}.
	\item $\pi_m$ is a vector of size $m$ which corresponds to the probability distribution over interpretations $\I_1, \dots, \I_m$.
\end{itemize}

\todo[inline]{Do I need to define \textit{here} the solution of PGEL-SAT by the minimization problem? Isn't it better in the development chapter? }

% definition

% column generation algorithm

% example