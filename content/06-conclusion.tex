%!TeX root=../tese.tex

\chapter{Conclusion and future work}
\label{cap:conclusion}

Different studies focused on description logics with probabilistic reasoning. In particular, finding a logic that is both tractable and expressive enough became a problem of special interest in the field. 

This work presented a fragment of $\elpp$ with probabilistic capabilities called Probabilistic Graphic $\el$. It was shown that its satisfiability problem can be modeled as a linear program, using a column generation technique. Each column generation is an instance of a MaxSAT problem, which can be solved by representing the knowledge base as a weighted graph and finding a minimal cut. 

Also, all algorithms for solving $\pgelsat$ were implemented and its tractability was presented. For this solution, a theoretical upper bound was estimated and experimental analysis confirm the polynomial complexity of the run time.

Furthermore, studies and implementations from this work contributed for improving the theory for this algorithm of \citet{Fin2020}, which is under development.

Further research directions include finding tractable extensions of $\pgel$. Existential body axioms and conjunctive axioms could be approximated using probabilistic restrictions; thus, it is expected to achieve tractable approximate reasoning for a probabilistic $\elpp$. Also, conditional probabilities could be modeled using linear restrictions.