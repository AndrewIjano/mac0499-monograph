%!TeX root=../tese.tex

\chapter{Introduction}
\label{cap:introduction}
\todo[inline]{Need to rewrite, but the example is OK}
Description logics are a family of formal knowledge representation languages, being of particular importance in providing a logical formalism for ontologies and the Semantic Web. Also, they are notable in biomedical informatics for assisting the codification of biomedical knowledge. Due to these uses, there is a great demand to find tractable (i.e., polynomial-time decidable) description logics.

One of them, the logic $\elpp$, is one of the most expressive description logics in which the complexity of inferential reasoning is tractable \citep{Baader2005a}. Even though it is expressive enough to deal with several practical applications, there was also a need to model uncertain knowledge.

\begin{example}
  \label{exmp:real-example}
Consider a medical situation, in which a patient may have non-specific symptoms, such as high fever, cough, and headache. Also, COVID-19, a severe acute respiratory syndrome caused by the  SARS-CoV-2 virus, is a disease that can account for those symptoms, but not all patients present all symptoms. Such an uncertain situation is suitable for probabilistic modeling.

In a certain hospital, a patient with a high fever has some probability of having COVID-19, but that probability is 20\% larger if the patient has a cough too. On the other hand, COVID-19 is not very prevalent and is not observed in the hospital 90\% of the time. If those probabilistic constraints are satisfiable, one can also ask the minimum and maximum probability that a hospital patient Mary, with fever and cough, is a suspect of suffering from COVID-19.
\end{example}

For classical propositional formulas, this problem, called \emph{probabilistic satisfiability} (PSAT), has already been presented with tractable fragments \citep{andersen2001easy}. On the other hand, in description logics, most studies result in intractable reasoning; moreover, by adding probabilistic reasoning capabilities to $\elpp$, to model such situation, the complexity reaches NP-completeness \citep{Fin2019b}.

To solve this problem, probabilistic constraints can be applied to axioms and its probabilistic satisfaction can be seen in a linear algebraic view. Furthermore, it can be reduced to an optimization problem, which can be solved by an adaptation of the simplex method with column generation \citep{Fin2019b}. Thus, it is possible to reduce the column generation problem to the \emph{weighted partial maximum satisfiability}.

Moreover, recent studies show that it is necessary to focus on a fragment of $\elpp$ for obtain tractable probabilistic reasoning, which will be called {Graphic} $\elpp$ ($\gelpp$) \citep{Fin2020}. Therefore, this fragment allows axioms to be seen as edges in a graph, as opposed to hyperedges in a hypergraph, which is the case of $\elpp$. This allows the use of graph-based machinery to develop a tractable algorithm for the \emph{weighted partial Maximum SATisfiability} for $\gelpp$ (Max $\gelpp$-SAT) and, as a result, a tractable probabilistic description logic.

\section{Objective}
Then, the objective of this project is to propose and implement tractable algorithms for weighted partial Max-SAT and Probabilistic SAT for a fragment of $\elpp$ description logic.

\section{Structure}
In this paper, we describe the implementation of these algorithms \footnote{Available at https://github.com/AndrewIjano/pgel-sat} and is organized as follows: Section \ref{sec:relatedWork} highlights related results in the literature. The basic definition of $\gelpp$ with its algorithms for MaxSAT and PSAT are described in Section \ref{sec:methods} and followed by Section \ref{sec:results}, which presents details about the implementation and its experimental evaluation.
