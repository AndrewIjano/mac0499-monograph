%!TeX root=../tese.tex

\chapter{Introduction}
\label{cap:introduction}
Description logics are a family of formal knowledge representation languages, being of particular importance in providing a logical formalism for ontologies and the Semantic Web. Also, they are notable in biomedical informatics for assisting the codification of biomedical knowledge. Due to these uses, there is a great demand to find tractable (i.e., polynomial-time decidable) description logics.

One of them, the logic $\elpp$, is one of the most expressive description logics in which the complexity of inferential reasoning is tractable \citep{Baader2005a}. Even though it is expressive enough to deal with several practical applications, there was also a need to model situations in which an axiom is not always true, which has already been proposed in the literature \citep{boole1854investigation}.

\begin{example}
  \label{exmp:real-example}
Consider a medical situation in which a patient may have non-specific symptoms, such as high fever, cough, and headache. Also, COVID-19, a severe acute respiratory syndrome caused by the  SARS-CoV-2 virus, is a disease that can account for those symptoms, but not all patients present all symptoms. Such an uncertain situation is suitable for probabilistic modeling.

In a certain hospital, a patient with a high fever has some probability of having COVID-19, but that probability is 20\% larger if the patient has a cough too. On the other hand, COVID-19 is not very prevalent and is not observed in the hospital 90\% of the time. If those probabilistic constraints are satisfiable, one can also ask the minimum and maximum probability that a hospital patient Mary, with fever and cough, is a suspect of suffering from COVID-19.
\end{example}

For classical propositional formulas, this problem, called \emph{probabilistic satisfiability} (PSAT), has already been presented with tractable fragments \citep{andersen2001easy}. On the other hand, in description logics, most studies result in intractable reasoning; moreover, by adding probabilistic reasoning capabilities to $\elpp$, to model such situation, the complexity reaches NP-completeness \citep{Fin2019b}. 

To solve this problem, ongoing studies of \citet{Fin2020} propose a fragment of $\elpp$ called \emph{Graphic $\el$} ($\gel$), which does not allow axioms with conjunctions or existential restrictions in the left-hand side. With this logic, using probabilistic constraints applied to axioms, its probabilistic satisfaction can be seen in a linear algebraic view. Furthermore, it can be reduced to an optimization problem, which can be solved by an adaptation of the simplex method with column generation. Thus, it is possible to reduce the column generation problem to the \emph{weighted partial maximum satisfiability for $\gel$} ($\gel$-MaxSAT). Another property of this fragment is that axioms can be modeled as edges in a graph, as opposed to hyperedges in a hypergraph, which is the case of $\elpp$. This allows the use of graph-based machinery to develop a tractable algorithm for the $\gel$-MaxSAT and, as a result, a tractable probabilistic description logic.

\section{Objectives}
The main objective of this work is contribute to the studies of \citet{Fin2020} in order to find and implement algorithms for a tractable probabilistic description logic. This goal can be summarized in the following points: 

\begin{enumerate}
  \item Investigate a potentially tractable fragment of $\elpp$;
  \item Study and implement tractable algorithms for the problem of $\gel$-MaxSAT;
  \item Study and implement algorithms for the problem of \emph{probabilistic satisfiability for $\gel$} ($\pgelsat$), using the $\gel$-MaxSAT solver as a subroutine. Thus, it is excepted to achieve a tractable algorithm for a probabilistic description logic.
\end{enumerate}

\section{Structure of this work}
This work is organized as follows. \cref{cap:background} presents the theoretical background about description logics and the logical fragment of interest. \cref{cap:development} describe the development of a tractable algorithm for solving probabilistic satisfiability in description logics. \cref{cap:experiments} presents experiments to analyse the tractability of this algorithms and discusess its results. After that, \cref{cap:relatedwork} revises the current studies about this topic. Then, \cref{cap:conclusion} summarizes the achievements and discusses further research. 