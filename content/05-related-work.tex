%!TeX root=../tese.tex

\chapter{Related work}
\label{cap:relatedwork}

The problem of probabilistic reasoning and extensions in logics to deal with uncertainty have been studied for several decades. The first known proposal of PSAT, for propositional formulas, is attributed to \citet{boole1854investigation} and it has already been shown to be NP-Complete \citep{georgakopoulos1988probabilistic}.

In the relational domain, the literature contain several logics with probabilistic reasoning capabilities although they have led to intractable decision problems.  Some of them extend the already intractable $\mathcal{ALC}$, with probabilistic constrains over concepts \citep{heinsohn1994probabilistic, lukasiewicz2008expressive, GutierrezBasultoEA11}. The probablistic Datalog$^\pm$ \citep{gottlob2013query}, that combines Datalog$^\pm$ with Markov logic networks \citep{richardson2006markov}, has a query entailment in PP-hard \citep{ceylan_datalog_compl_2016}. For the expressive and lightweight $\el$-family, some extensions such as \citet{gutierrez2017probabilistic,ceylan2017bayesian} have led to \textsc{ExpTime}-hard or PP-complete probabilistic reasoning; futhermore, NP-completeness can be achieved with probability capabilities over axioms \citep{Fin2019b}.

On the other hand, tractable probabilistic reasoning was achieved by different ways. \citet{Domingos_Webb_2012} proposed the first tractable first-order probabilistic logic, called Tractable Markov Logic (TML), by imposing an hierarchical structure on its domain. A second aproach from TML led to a straightforward handling of existence uncertainty \citep{webb2013tractable}. Besides that, a probabilistic extension of DL-Lite based on Bayesian networks allows satisfiability checking in \textsc{LogSpace} \citep{damato_prob_dl_lite_2008}. 

Furthermore, some studies implies that the research on MaxSAT has a impact on the solutions of PSAT problems \citep{andersen2001easy}. Also, there was already proposed a MaxSAT-solver for a propositional fragment of horn logic by a max-flow/min-cut formulation \citep{jaumard1987complexity}, which inspired the solution described in this study.