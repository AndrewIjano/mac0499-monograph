%!TeX root=../tese.tex

\chapter{Related work}
\label{cap:relatedwork}

The problem of probabilistic reasoning and extensions in logics to deal with uncertainty have been studied for several decades. The first known proposal of PSAT, for propositional formulas, is attributed to \citet{boole1854investigation} and it has already been shown to be NP-Complete \citep{georgakopoulos1988probabilistic}.

In the relational domain, the literature contain several logics with probabilistic reasoning capabilities although they have led to intractable decision problems.  Some of them extend the already intractable $\mathcal{ALC}$, with probabilistic constrains over concepts \citep{heinsohn1994probabilistic, lukasiewicz2008expressive, GutierrezBasultoEA11}. For the expressive and lightweight $\el$-family, some extensions such as \citet{gutierrez2017probabilistic,ceylan2017bayesian} have led to \textsc{ExpTime}-hard or PP-complete probabilistic reasoning; futhermore, NP-completeness can be achieved with probability capabilities over axioms \citep{Fin2019b}.

On the other hand, many results implies that the research on Max-SAT has a impact on the solutions of PSAT problems \citep{andersen2001easy}. Also, there was already proposed a MaxSAT-solver for a propositional fragment of horn logic by a max-flow/min-cut formulation \citep{jaumard1987complexity}. Thus, it is expected to ask if one could also take such results to a relational domain.
