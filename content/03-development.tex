%!TeX root=../tese.tex

\chapter{Development}
\label{cap:development}

In this section, we describe the development of a tractable algorithm for PGEL-SAT\footnote{Available at \url{https://github.com/AndrewIjano/pgel-sat}.}. It was built using Python programming language \citep{python3}.

\section{Input and output format}

The algorithm accepts as input a $\pgel$ KB encoded in an OWL 2 ontology. Both certain and uncertain knowledge must be a $\gel$-CBox in the normal form, with the additional support of equivalence axioms. Due to limitations in the OWL parser used, which will be detailed further, RDF/XML and NTriples are the only file formats supported.

In addition, uncertain axioms must have an annotation (\texttt{rdfs:comment}) with its unique numerical index. That is, given an uncertain axiom $Ax_i$, its annotation must be of the following form
\begin{lstlisting}[style=mystyle]
	#! pbox-id
	$i$
\end{lstlisting}
\todo{I dont know how to place correctly this code}

PBox restrictions are represented with annotations in the $\top$ concept (\texttt{owl:Thing}). That is, given a restriction of the form $a_0 P(Ax_0) + a_1P(Ax_1) + \cdots + a_{n-1} P(Ax_{n-1}) = b$, its annotation must be of the form
\begin{lstlisting}[style=mystyle]
    #! pbox-restriction
    0   $a_0$
    1   $a_1$
    $\cdots$
    $n-1$ $a_{n-1}$
    ==
    $b$
\end{lstlisting}

The output of the algorithm is \texttt{True} if the given KB has a solution, and \texttt{False} if not.

\improvement{Maybe explain how to execute the algorithm}

\section{OWL parser}
- lib used: owlready2

- explores the data Structure of this lib to get the right information

\section{Knowledge Base}
- how it is stored


\section{GEL-MaxSAT}
- explain the theory of the solution

- pseudocode of the algorithm 

- max flow / min cut

\section{Linear solver}
- explain the theory of the solution

  - define cost vector (and ill-formed and well-formed columns)
  
  - define the linear solution from the cost vector and the linear view of the PGEL-SAT problem

- explain the implementation; lib used: glpk python

- interior points -> tractable

\section{PGEL-SAT reasoner}
- wrap all the concepts

- column generation

- numpy

- explain the final solution
