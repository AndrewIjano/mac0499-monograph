%%%%%%%%%%%%%%%%%%%%%%%%%%%%%%%%%%%%%%%%%%%%%%%%%%%%%%%%%%%%%%%%%%%%%%%%%%%%%%%%
%%%%%%%%%%%%%%%%%%%%%%%%%% HIPERLINKS E REFERÊNCIAS %%%%%%%%%%%%%%%%%%%%%%%%%%%%
%%%%%%%%%%%%%%%%%%%%%%%%%%%%%%%%%%%%%%%%%%%%%%%%%%%%%%%%%%%%%%%%%%%%%%%%%%%%%%%%

% O comando \ref por padrão mostra apenas o número do elemento a que se
% refere; assim, é preciso escrever "veja a Figura~\ref{grafico}" ou
% "como visto na Seção~\ref{sec:introducao}". Usando o pacote hyperref
% (carregado mais abaixo), esse número é transformado em um hiperlink.
%
% Se você quiser mudar esse comportamento, ative as packages varioref
% e cleveref e também as linhas "labelformat" e "crefname" mais abaixo.
% Nesse caso, você deve escrever apenas "veja a \ref{grafico}" ou
% "como visto na \ref{sec:introducao}" etc. e o nome do elemento será
% gerado automaticamente como hiperlink.
%
% Se, além dessa mudança, você quiser usar os recursos de varioref ou
% cleveref, mantenha as linhas labelformat comentadas e use os comandos
% \vref ou \cref, conforme sua preferência, também sem indicar o nome do
% elemento, que é inserido automaticamente. Vale lembrar que o comando
% \vref de varioref pode causar problemas com hyperref, impedindo a
% geração do PDF final.
%
% ATENÇÃO: varioref, hyperref e cleveref devem ser carregadas nessa ordem!
\usepackage{varioref}

%\labelformat{figure}{Figura~#1}
%\labelformat{table}{Tabela~#1}
%\labelformat{equation}{Equação~#1}
%% Isto não funciona corretamente com os apêndices; o comando seguinte
%% contorna esse problema
%%\labelformat{chapter}{Capítulo~#1}
%\makeatletter
%\labelformat{chapter}{\@chapapp~#1}
%\makeatother
%\labelformat{section}{Seção~#1}
%\labelformat{subsection}{Seção~#1}
%\labelformat{subsubsection}{Seção~#1}

% XMP (eXtensible Metadata Platform) is a mechanism proposed by Adobe for
% embedding document metadata within the document itself. The package
% integrates seamlessly with hyperref and requires virtually no modifications
% to documents that already exploit hyperref's mechanisms for specifying PDF
% metadata. It should be loaded before hyperref.
\usepackage{hyperxmp}

% HACK ALERT! hyperxmp usa atenddvi, que tem este bug:
% https://github.com/ho-tex/atenddvi/issues/1 . Aparentemente, ele não
% afeta xelatex (cf https://github.com/latex3/latex2e/issues/94 ), então
% só precisamos nos preocupar com pdflatex e lualatex. Com pdflatex,
% hyperxmp não deveria utilizar atenddvi, mas às vezes usa. Com lualatex,
% me parece que atenddvi também não é necessária, pois hyperxmp não usa
% \special ou \write com lualatex. Então, vamos deixar hyperxmp carregar
% atenddvi mas (1) vamos impedi-la de funcionar e (2) vamos garantir que
% hyperxmp sempre use AtEndDocument com pdflatex e lualatex. Há ainda um
% outro truque para contornar esse problema na configuração da bibliografia,
% mas não podemos ter certeza que apenas a bibliografia pode ser afetada
% por este bug.

% TODO: remover isto quando o bug for corrigido
% \makeatletter
% \ifpdf
%   \ifxetex
%     \relax
%   \else
%     \let\AtEndDvi@AtBeginShipout\relax
%     \let\AtEndDvi@CheckImpl\relax
%     \let\hyxmp@at@end\AtEndDocument
%   \fi
% \fi
% \makeatother

% Usamos "PassOptions" aqui porque outras packages definem opções para
% hyperref também e chamar a package com opções diretamente gera conflitos.
\PassOptionsToPackage{
  unicode=true,
  plainpages=false,
  pdfpagelabels,
  colorlinks=true,
  %citecolor=black,
  %linkcolor=black,
  %urlcolor=black,
  %filecolor=black,
  citecolor=DarkGreen,
  linkcolor=NavyBlue,
  urlcolor=DarkRed,
  filecolor=green,
  bookmarksopen=true,
  % hyperref não gera hyperlinks corretos em índices remissivos criados com
  % xindy; assim, é possível desabilitar essa função aqui e gerar os
  % hyperlinks através da configuração de xindy definida anteriormente. Com
  % makeindex (o default), quem precisa criar os hyperlinks é hyperref.
  %hyperindex=false,
}{hyperref}

% Cria hiperlinks para capítulos, seções, \ref's, URLs etc.
\usepackage{hyperref}

\usepackage[nameinlink,noabbrev,capitalise]{cleveref}
%% cleveref não tem tradução para o português
%\crefname{figure}{Figura}{Figuras}
%\crefname{table}{Tabela}{Tabelas}
\crefname{chapter}{Chapter}{Chapters}
%\crefname{section}{Seção}{Seções}
%\crefname{subsection}{Seção}{Seções}
%\crefname{subsubsection}{Seção}{Seções}
%\crefname{appendix}{Apêndice}{Apêndices}
%\crefname{subappendix}{Apêndice}{Apêndices}
%\crefname{subsubappendix}{Apêndice}{Apêndices}
%\crefname{line}{Linha}{Linhas}
%\crefname{subfigure}{Figura}{Figuras}
%\crefname{equation}{Equação}{Equações}
%\crefname{listing}{Código-fonte}{Códigos-fonte}
%\crefname{lstlisting}{Código-fonte}{Códigos-fonte}
%\crefname{lstnumber}{Linha}{Linhas}
%\crefrangelabelformat{chapter}{#3#1#4~a~#5#2#6}
%\crefrangelabelformat{section}{#3#1#4~a~#5#2#6}
%\newcommand{\crefrangeconjunction}{ e }
%\newcommand{\crefpairconjunction}{ e }
%\newcommand{\crefmiddleconjunction}{, }
%\newcommand{\creflastconjunction}{ e }
%\crefmultiformat{type}{first}{second}{middle}{last}
%\crefrangemultiformat{type}{first}{second}{middle}{last}

% ao criar uma referência hyperref para um float, a referência aponta
% para o final do caption do float, o que não é muito bom. Este pacote
% faz a referência apontar para o início do float (é possível personalizar
% também). Esta package é incompatível com a classe beamer (usada para
% criar posters e apresentações), então testamos a compatibilidade antes
% de carregá-la.
\ifboolexpr{
  test {\ifcsdef{figure}} and
  test {\ifcsdef{figure*}} and
  test {\ifcsdef{table}} and
  test {\ifcsdef{table*}}
}{\usepackage[all]{hypcap}}{}

% hyperref detecta url's definidas com \url que começam com "http" e
% "www" e cria links adequados. No entanto, quando a url não começa
% com essas strings (por exemplo, "usp.br"), hyperref considera que
% se trata de um link para um arquivo local. Isto força todas as
% \url's que não tem esquema definido a serem do tipo http.
\hyperbaseurl{http://}
