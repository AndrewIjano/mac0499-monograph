% Arquivo LaTeX de exemplo de dissertação/tese a ser apresentada à CPG do IME-USP
%
% Criação: Jesús P. Mena-Chalco
% Revisão: Fabio Kon e Paulo Feofiloff
% Adaptação para UTF8, biblatex e outras melhorias: Nelson Lago
%
% Except where otherwise indicated, these files are distributed under
% the MIT Licence. The example text, which includes the tutorial and
% examples as well as the explanatory comments in the source, are
% available under the Creative Commons Attribution International
% Licence, v4.0 (CC-BY 4.0) - https://creativecommons.org/licenses/by/4.0/


%%%%%%%%%%%%%%%%%%%%%%%%%%%%%%%%%%%%%%%%%%%%%%%%%%%%%%%%%%%%%%%%%%%%%%%%%%%%%%%%
%%%%%%%%%%%%%%%%%%%%%%%%%%%%%%% PREÂMBULO LaTeX %%%%%%%%%%%%%%%%%%%%%%%%%%%%%%%%
%%%%%%%%%%%%%%%%%%%%%%%%%%%%%%%%%%%%%%%%%%%%%%%%%%%%%%%%%%%%%%%%%%%%%%%%%%%%%%%%
\documentclass[12pt,twoside,brazil,english]{book}
\usepackage[a4paper]{geometry}

\geometry{
  %top=32mm,
  %bottom=28mm,
  %left=24mm,
  %right=34mm,
  textwidth=152mm, % 210-24-34
  textheight=237mm, % 297-32-28
  vmarginratio=8:7, % 32:28
  hmarginratio=12:17, % 24:34
  % Com geometry, esta medida não é tão relevante; basta garantir que ela
  % seja menor que "top" e que o texto do cabeçalho caiba nela.
  headheight=25.4mm,
  % distância entre o início do texto principal e a base do cabeçalho;
  % ou seja, o cabeçalho "invade" a margem superior nessa medida. Essa
  % é a medida que determina a posição do cabeçalho
  headsep=11mm,
  footskip=10mm,
  marginpar=20mm,
  marginparsep=5mm,
}

% Vários pacotes e opções de configuração genéricos; para personalizar o
% resultado, modifique estes arquivos.
\input{extras/basics}
\input{extras/fonts}
\input{extras/floats}
\input{extras/thesis-formatting}
\input{extras/index}
\input{extras/hyperlinks}
\input{extras/source-code}
%%%%%%%%%%%%%%%%%%%%%%%%%%%%%%%%%%%%%%%%%%%%%%%%%%%%%%%%%%%%%%%%%%%%%%%%%%%%%%%%
%%%%%%%%%%%%%%%%%%%%%%%%%%%% OUTROS PACOTES ÚTEIS %%%%%%%%%%%%%%%%%%%%%%%%%%%%%%
%%%%%%%%%%%%%%%%%%%%%%%%%%%%%%%%%%%%%%%%%%%%%%%%%%%%%%%%%%%%%%%%%%%%%%%%%%%%%%%%

% Você provavelmente vai querer ler a documentação de alguns destes pacotes
% para personalizar algum aspecto do trabalho ou usar algum recurso específico.

% A classe Book inclui o comando \appendix, que (obviamente) permite inserir
% apêndices no documento. No entanto, não há suporte similar para anexos. Esta
% package (que não é padrão do LaTeX, foi criada para este modelo) define o
% comando \annex. Ela deve ser carregada depois de hyperref.
\dowithsubdir{extras/}{\usepackage{annex}}

% Para inserir separações no texto que não correspondem a seções com um nome
% definido, é comum usar um ornamento ou florão (em inglês e francês: fleuron).
% Esta package define o comando \frufru que insere um florão desse tipo.
\dowithsubdir{extras/}{\usepackage{frufru}}

% Formatação personalizada das listas "itemize", "enumerate" e
% "description", além de permitir criar novos tipos de listas.
% Com a opção "inline", a package define os novos ambientes "itemize*",
% "description*" e "enumerate*", que fazem os itens da lista como
% parte de um único parágrafo. Como ela causa problemas com
% beamer, apenas a carregamos se não estivermos usando beamer.
\makeatletter
\@ifclassloaded{beamer}
  {}
  {\usepackage[inline]{enumitem}}
\makeatother

% Sublinhado e outras formas de realce de texto
\usepackage{soul}
\usepackage{soulutf8}

% Melhorias e personalização do sublinhado com soul (comando \ul)

% Distância e largura do sublinhado
\setul{1.4pt}{.5pt}

% btul -> "Better Underline" (https://alexwlchan.net/2017/10/latex-underlines/ )
% Sublinhado sem cruzar as linhas descendentes dos caracteres
\usepackage[outline]{contour}
\contourlength{1.1pt}
\let\ORIGul\ul
\newcommand{\btul}[2][white]{%
  \contourlength{1.1pt}%
  \setul{1.4pt}{.5pt}%
  \ORIGul{{\phantom{#2}}}% Faz o sublinhado; precisa das chaves adicionais!
  \llap{\contour{#1}{#2}}% Escreve o texto com fundo branco/colorido
}

% Notação bra-ket
%\usepackage{braket}

% Vários recursos para apresentação de números e grandezas (unidades, notação
% científica, melhor apresentação de números longos etc.), além de permitir
% alinhar números em tabelas pelo ponto decimal (como a package dcolumn)
% através do tipo de coluna "S". Por exemplo, \SI{10}{\hertz} ou
% \num[round-mode=places,round-precision=2]{3.1415926}.
\usepackage[binary-units]{siunitx}
\sisetup{
  mode=text,
  round-mode=places,
}

\providetranslation[to=Portuguese]{to (numerical range)}{a}
\providetranslation[to=Portuguese]{and}{e}
\addto\extrasbrazil{\sisetup{output-decimal-marker = {,}}}

% Citações melhores; se você pretende fazer citações de textos
% relativamente extensos, vale a pena ler a documentação. biblatex
% utiliza recursos deste pacote.
\usepackage{csquotes}

\usepackage{url}
% URL com fonte sem serifa ao invés de teletype
\urlstyle{sf}

% Permite inserir comentários, muito bom durante a escrita do texto;
% você também pode se interessar pela package pdfcomment.
\usepackage[textsize=scriptsize,colorinlistoftodos,textwidth=2.5cm]{todonotes}
\presetkeys{todonotes}{color=orange!40!white}{}

% Comando para fazer notas com highlight no texto correspondente:
% \hltodo[texto][opções]{comentário}
\makeatletter
\if@todonotes@disabled
  \NewDocumentCommand{\hltodo}{O{} O{} +m}{#1}
\else
  \NewDocumentCommand{\hltodo}{O{} O{} +m}{
    \ifstrempty{#1}{}{\texthl{#1}}%
    \todo[#2]{#3}{}%
  }
\fi
\makeatother

% Vamos reduzir o espaçamento entre linhas nas notas/comentários
\makeatletter
\xpatchcmd{\@todo}
  {\renewcommand{\@todonotes@text}{#2}}
  {\renewcommand{\@todonotes@text}{\begin{spacing}{0.5}#2\end{spacing}}}
  {}
  {}
\makeatother

% Símbolos adicionais, como \celsius, \ohm, \perthousand etc.
%\usepackage{gensymb}

% Permite criar listas como glossários, listas de abreviaturas etc.
% https://en.wikibooks.org/wiki/LaTeX/Glossary
%\usepackage{glossaries}

% Permite formatar texto em colunas
\usepackage{multicol}

% Gantt charts; útil para fazer o cronograma para o exame de
% qualificação, por exemplo.
\usepackage{pgfgantt}

% Na versão 5 do pacote pgfgantt, a opção "compress calendar"
% deixou de existir, sendo substituída por "time slot unit=month".
% Aqui, um truque para funcionar com ambas as versões.
\makeatletter
\@ifpackagelater{pgfgantt}{2018/01/01}
  {\ganttset{time slot unit=month}}
  {\ganttset{compress calendar}}
\makeatother

% Estes parâmetros definem a aparência das gantt charts e variam
% em função da fonte do documento.
\ganttset{
    time slot format=isodate-yearmonth,
    vgrid,
    x unit=1.7em,
    y unit title=3ex,
    y unit chart=4ex,
    % O "strut" é necessário para alinhar o baseline dos nomes dos meses
    title label font=\strut\footnotesize,
    group label font=\footnotesize\bfseries,
    bar label font=\footnotesize,
    milestone label font=\footnotesize\itshape,
    % "align=right" é necessário para \ganttalignnewline funcionar
    group label node/.append style={align=right},
    bar label node/.append style={align=right},
    milestone label node/.append style={align=right},
    group incomplete/.append style={fill=black!50},
    bar/.append style={fill=black!25,draw=black},
    bar incomplete/.append style={fill=white,draw=black},
    % Não é preciso imprimir "0%"
    progress label text=\ifnumequal{#1}{0}{}{(#1\%)},
    % Formato e tamanho dos elementos
    title height=.9,
    group top shift=.4,
    group left shift=0,
    group right shift=0,
    group peaks tip position=0,
    group peaks width=.2,
    group peaks height=.3,
    milestone height=.4,
    milestone top shift=.4,
    milestone left shift=.8,
    milestone right shift=.2,
}

% Em inglês, tanto o nome completo quanto a abreviação do mês de maio
% são "May"; por conta disso, na tradução em português LaTeX erra a
% abreviação. Como talvez usemos o nome inteiro do mês em outro lugar,
% ao invés de forçar a tradução para "Mai" globalmente, fazemos isso
% apenas em ganttchart.
\AtBeginEnvironment{ganttchart}{\deftranslation[to=Portuguese]{May}{Mai}}

% Ilustrações, diagramas, gráficos etc. criados diretamente em LaTeX.
% Também é útil se você quiser importar gráficos gerados com GnuPlot.
\usepackage{tikz}

% Gráficos gerados diretamente em LaTeX; é possível usar tikz para
% isso também.
\usepackage{pgfplots}
% sobre níveis de compatibilidade do pgfplots, veja
% https://tex.stackexchange.com/a/81912/183146
%\pgfplotsset{compat=1.14} % TeXLive 2016
%\pgfplotsset{compat=1.15} % TeXLive 2017
%\pgfplotsset{compat=1.16} % TeXLive 2019
\pgfplotsset{compat=newest}

% Importação direta de arquivos gerados por gnuplot com o
% driver/terminal "lua tikz"; esta package não faz parte da
% instalação padrão do LaTeX, mas sim do gnuplot.
%\usepackage{gnuplot-lua-tikz}

% O formato pdf permite anexar arquivos ao documento, que aparecem
% na página como ícones "clicáveis"; esta package implementa esse
% recurso em LaTeX.
%\usepackage{attachfile}


% EL
\newcommand{\el}{\mathcal{EL}} 
% EL++
\newcommand{\elpp}{\mathcal{EL}^{++}}
% GEL++
\newcommand{\gelpp}{\mathcal{GEL}^{++}}
% PGEL
\newcommand{\pgel}{\text{P}\gelpp}
% PGEL-SAT
\newcommand{\pgelsat}{\text{PGEL-SAT}}
% concrete domain D
\newcommand{\D}{\mathcal{D}}
% concept names
\newcommand{\Nc}{\mathsf{N_C}}
% role names
\newcommand{\Nr}{\mathsf{N_R}}
% individual names
\newcommand{\Ni}{\mathsf{N_I}}
% Nw
\newcommand{\Nw}{\mathsf{N}_w}
% Ew
\newcommand{\Ew}{\mathsf{E}_w}
% interpretation
\newcommand{\I}{\mathcal{I}}
% CBox C
\newcommand{\CC}{\mathcal{C}}
% Soft CBox Z
\newcommand{\Z}{\mathcal{Z}}
% basic concepts
\newcommand{\bc}{\mathsf{BC}_\mathcal{C}}
% P
\newcommand{\PP}{\mathcal{P}}
% Knowledge base
\newcommand{\CP}{\langle \C, \PP \rangle}
% is a
\newcommand{\isa}{\sqsubseteq}
% C is a D
\newcommand{\CisaD}{C \sqsubseteq D}
% knowledge base
\newcommand{\K}{\mathcal{K}}

% knowledge base
\newcommand{\cbox}{\mathcal{C}}
% knowledge base
\newcommand{\abox}{\mathcal{A}}

% todo package
\usepackage{todonotes}

% for tables
\usepackage{booktabs}
\renewcommand{\arraystretch}{1.3} 

% inline enumerate
\usepackage[inline]{enumitem}

% example environment
\usepackage{aliascnt}
\usepackage[framemethod=tikz]{mdframed}
\newcounter{example}
\newenvironment{example}
  {%
  \refstepcounter{example}
  \par\noindent\normalfont\textbf{Example~\theexample}\par\nopagebreak%
  \begin{mdframed}[
     linewidth=1pt,
     linecolor=gray,
     bottomline=false, 
     topline=false,
     rightline=false,
     innerrightmargin=0pt,
     innertopmargin=0pt,
     innerbottommargin=0pt,
     innerleftmargin=1em,% Distance between vertical rule & proof content
     skipabove=.5\baselineskip
   ]}
  {\end{mdframed}}
\numberwithin{example}{chapter}
\providecommand*{\exampleautorefname}{Example}


% Diretórios onde estão as figuras; com isso, não é preciso colocar o caminho
% completo em \includegraphics (e nem a extensão).
\graphicspath{{img/}}

% Comandos rápidos para mudar de língua:
% \en -> muda para o inglês
% \br -> muda para o português
% \texten{blah} -> o texto "blah" é em inglês
% \textbr{blah} -> o texto "blah" é em português
\babeltags{br = brazil, en = english}

% Espaçamento simples
\singlespacing


%%%%%%%%%%%%%%%%%%%%%%%%%%%%%%% BIBLIOGRAFIA %%%%%%%%%%%%%%%%%%%%%%%%%%%%%%%%%%%

\usepackage[
  % \citet, \citep
  natbib=true,
  % similar a plainnat (autor-data)
  style=extras/plainnat-ime,
]{biblatex}

\addbibresource{bibliography.bib}
\input{extras/bibconfig}

%%%%%%%%%%%%%%%%%%%%%%% METADADOS (TÍTULO, AUTOR ETC.) %%%%%%%%%%%%%%%%%%%%%%%%%

\title{Tractable Probabilistic Description Logic}
\author{Andrew Ijano Lopes}

% TODO: update keywords
\hypersetup{
  pdfkeywords={LaTeX, tese, dissertação, IME/USP},
}

% Este pacote define o formato sugerido da capa, páginas de rosto,
% dedicatória e resumo. Se você pretende criar essas páginas manualmente,
% não precisa carregar este pacote nem carregar o arquivo folhas-de-rosto.
\dowithsubdir{extras/}{\usepackage{imeusp-capa}}

%%%%%%%%%%%%%%%%%%%%%%%%%%%%%%%%%%%%%%%%%%%%%%%%%%%%%%%%%%%%%%%%%%%%%%%%%%%%%%%%
%%%%%%%%%%%%%%%%%%%%%%% AQUI COMEÇA O CONTEÚDO DE FATO %%%%%%%%%%%%%%%%%%%%%%%%%
%%%%%%%%%%%%%%%%%%%%%%%%%%%%%%%%%%%%%%%%%%%%%%%%%%%%%%%%%%%%%%%%%%%%%%%%%%%%%%%%

\begin{document}

  % Aqui vai o conteúdo inicial que aparece antes do capítulo 1, ou seja,
  % página de rosto, resumo, sumário etc. O comando frontmatter faz números
  % de página aparecem em algarismos romanos ao invés de arábicos e
  % desabilita a contagem de capítulos.
  \frontmatter

  % Este formato está (re)definido na package imeusp-headers
  \pagestyle{plain}


  %%%%%%%%%%%%%%%%%%%%%%%%%%% CAPA E FOLHAS DE ROSTO %%%%%%%%%%%%%%%%%%%%%%%%%%%%%

  % Para gerar o título sem seguir o formato deste modelo, você pode usar o
  % comando padrão do LaTeX "\maketitle".
  %\maketitle

  % Capa e folhas de rosto no formato sugerido para teses/dissertações do IME/USP.
  % Se for gerar a capa etc. manualmente, remova.
  \onehalfspacing % Espaçamento 1,5 nas páginas iniciais
  %\input{conteudo-exemplo/folhas-de-rosto}
  \input{front-back-matter/title-page.tex}
  
  \par

  %%%%%%%%%%%%%%%%%%%%%%%%%%%%%%%% CAPÍTULOS %%%%%%%%%%%%%%%%%%%%%%%%%%%%%%%%%%%%%

  % Aqui vai o conteúdo principal do trabalho, ou seja, os capítulos que compõem
  % a dissertação/tese. O comando mainmatter reinicia a contagem de páginas,
  % modifica a numeração para números arábicos e ativa a contagem de capítulos.
  \mainmatter

  % Este formato está definido na package imeusp-headers e só funciona com
  % book/report, pois usa o nome dos capítulos nos cabeçalhos.
  \pagestyle{mainmatter}

  % Espaçamento simples
  \singlespacing

  %!TeX root=../tese.tex

\chapter{Introduction}
\label{cap:introduction}
\improvement[inline]{Need to rewrite, but the example is OK}
Description logics are a family of formal knowledge representation languages, being of particular importance in providing a logical formalism for ontologies and the Semantic Web. Also, they are notable in biomedical informatics for assisting the codification of biomedical knowledge. Due to these uses, there is a great demand to find tractable (i.e., polynomial-time decidable) description logics.

One of them, the logic $\elpp$, is one of the most expressive description logics in which the complexity of inferential reasoning is tractable \citep{Baader2005a}. Even though it is expressive enough to deal with several practical applications, there was also a need to model uncertain knowledge.

\begin{example}
  \label{exmp:real-example}
Consider a medical situation, in which a patient may have non-specific symptoms, such as high fever, cough, and headache. Also, COVID-19, a severe acute respiratory syndrome caused by the  SARS-CoV-2 virus, is a disease that can account for those symptoms, but not all patients present all symptoms. Such an uncertain situation is suitable for probabilistic modeling.

In a certain hospital, a patient with a high fever has some probability of having COVID-19, but that probability is 20\% larger if the patient has a cough too. On the other hand, COVID-19 is not very prevalent and is not observed in the hospital 90\% of the time. If those probabilistic constraints are satisfiable, one can also ask the minimum and maximum probability that a hospital patient Mary, with fever and cough, is a suspect of suffering from COVID-19.
\end{example}

For classical propositional formulas, this problem, called \emph{probabilistic satisfiability} (PSAT), has already been presented with tractable fragments \citep{andersen2001easy}. On the other hand, in description logics, most studies result in intractable reasoning; moreover, by adding probabilistic reasoning capabilities to $\elpp$, to model such situation, the complexity reaches NP-completeness \citep{Fin2019b}.

To solve this problem, probabilistic constraints can be applied to axioms and its probabilistic satisfaction can be seen in a linear algebraic view. Furthermore, it can be reduced to an optimization problem, which can be solved by an adaptation of the simplex method with column generation \citep{Fin2019b}. Thus, it is possible to reduce the column generation problem to the \emph{weighted partial maximum satisfiability}.

Moreover, recent studies show that it is necessary to focus on a fragment of $\elpp$ for obtain tractable probabilistic reasoning, which will be called {Graphic} $\elpp$ ($\gelpp$) \citep{Fin2020}. Therefore, this fragment allows axioms to be seen as edges in a graph, as opposed to hyperedges in a hypergraph, which is the case of $\elpp$. This allows the use of graph-based machinery to develop a tractable algorithm for the \emph{weighted partial Maximum SATisfiability} for $\gelpp$ (Max $\gelpp$-SAT) and, as a result, a tractable probabilistic description logic.

\section{Objective}
Then, the objective of this project is to propose and implement tractable algorithms for weighted partial Max-SAT and Probabilistic SAT for a fragment of $\elpp$ description logic.

\section{Structure}
In this paper, we describe the implementation of these algorithms \footnote{Available at https://github.com/AndrewIjano/pgel-sat} and is organized as follows: Section \ref{sec:relatedWork} highlights related results in the literature. The basic definition of $\gelpp$ with its algorithms for MaxSAT and PSAT are described in Section \ref{sec:methods} and followed by Section \ref{sec:results}, which presents details about the implementation and its experimental evaluation.

  %!TeX root=../tese.tex

\chapter{Background}
\label{cap:background}

In this chapter, we present the theoretical background of description logics, 

\improvement[inline]{Add initial description of the chapter}

\section{Description logics}
Description logics (DLs) are used to represent knowledge, such as the semantic of words,  people and their relations, and medical terms. These scenarios require precise specification and meaning so that different systems behave the same way. The first DL modeling languages appeared in the mid-1980s and have an important role in the context of the Semantic Web, an initiative to represent web content in a form that is more machine friendly \citep{krotzsch2012description}.

As their name suggests, DLs are logics; indeed, most of them are fragments of first-order logic. This relation with logics is what provides their precise specification, called \emph{formal semantics}. Also, it equips their languages with a formal deduction to \emph{infer} additional information, and the computation of these inferences is called \emph{reasoning}. The performance of algorithms for reasoning strongly relies on the expressiveness of the DL: fast algorithms usually exist for lightweight logics. Then, there is not just a single DL because the balance between expressiveness and performance depends on the application. \citep{krotzsch2012description}

\subsection{Building blocks of description logic ontologies}
A DL is composed of concepts, roles, and individual names. Concepts are sets of individuals, roles are binary relations between individuals and individual names are single individuals in the domain.

For example, an ontology\todo{Maybe explain what is an ontology} modeling the situation in \autoref{exmp:real-example} can use the concepts \textsf{Patient}, to represent the set of all patients in the hospital, and \textsf{Symptom}, to represent the set of all symptoms; roles such \textsf{hasSymptom}, to represent the binary relation between patients and symptoms; and individual names such as \textsf{mary} and \textsf{s1}, to represent the individuals Mary and Mary's symptoms.

Additionally, DLs allows us to describe more complex situations, creating new concepts and roles from the previously defined ones. 

Some concept constructors provide boolean operations similar to that found in set theory and logic expressions. For example, if we want to describe the set of individuals that are both fever and cough, we could use the \emph{conjunction} operator, as follows
\[
	\textsf{Fever} \sqcap \textsf{Cough}.
\]

We can link concepts and roles using role restrictions. For example, to describe all individuals that are suspect of some disease that is COVID-19, we use the \emph{existential restriction}
\[
	\exists \textsf{suspectOf}.\textsf{COVID-19}.
\]

Also, to define concepts with only one individual we use \emph{nominals} like $\{\textsf{mary}\}$.

More expressive logics can have other operations such as \emph{disjunction} ($C \sqcup D$), \emph{negation} ($\lnot C$), \emph{universal restriction} ($\forall r.C$) and \emph{number restrictions} ($\leq\!\! n\, r.C$).  

To capture knowledge about the world, DL ontologies also allow us to describe relations between concepts, roles, and individual names. For example, the fact that all fevers are symptoms is represented by the \emph{concept inclusion}
\[
	\textsf{Fever} \isa \textsf{Symptom};	
\]
the knowledge that someone that has a symptom which is caused by some disease is suspect of that disease can be expressed by the \emph{role inclusion} with a \emph{role composition}
\[
	\textsf{hasSymptom} \, \circ \, \textsf{hasCause} \isa \textsf{suspectOf};	
\]
and the fact that Mary is a patient of the hospital and has symptoms is represented by the \emph{assertions} $\textsf{Patient}(\textsf{mary})$ and $\textsf{hasSymptom(mary, s1)}$.

After that, if we have a set of these relations, one could ask if there is a set of individuals, or instances, that satisfies these relations, which is called an \emph{interpretation}. Interpretations can be understood as the assignment of meaning to logical terms in an ontology. Because a DL usually considers all the possibles situations, property that is sometimes referred to as \emph{open world assumption}, an ontology can have multiple satisfiable interpretations. The fewer restrictions it has, the more interpretations satisfy this ontology. The computational complexity to find the existence of these interpretations is one of the key aspects to choose different DL fragments.

These terms will be formally defined in the \autoref{sec:el}, in the case of the DL $\elpp$.

\subsection{Description logic fragments and OWL}

There are many DL fragments. Each subset of features, like those described previously, can lead to different fragments of first-order logic. For example, the logic $\mathcal{ALC}$ does not allow role inclusions and admits only $\sqcap, \sqcup, \lnot, \exists$ and $\forall$ as concept constructors; their best reasoning algorithms, however, are worst-case exponential time. On the other hand, the $\el$ logic allows only $\sqcap$ and $\exists$ as concept constructors, and its reasoning algorithms are polynomial time.

To express DL ontologies, the World Wide Web Consortium (W3C) designed the OWL~2 Web Ontology Language (OWL 2) \citep{owl2}. This declarative language is part of the W3C's Semantic Web technology stack and comes with various syntaxes, such as RDF/XML. Because of this use on the web, names in OWL are \emph{international resource identifiers} (IRIs).

\section{The description logic \texorpdfstring{$\elpp$}{����++}}
\label{sec:el}

$\elpp$ is an extension of the DL $\el$ \citep{Baader2005a}. It was created with large bio-health ontologies in mind, such as SNOMED-CT, the NCI thesaurus, and Galen, and became an official OWL 2 profile \citep{owl2}. We concentrate on presenting $\elpp$ without concrete domains.

\subsection{Syntax}
In $\elpp$, \emph{concept descriptions} are defined inductively from a set $\Nc$ of \emph{concept names}, a set $\Nr$ of \emph{role names} and set $\Ni$ of \emph{individual names} as follows:
\begin{itemize}
	\item $\top$, $\bot$ and concept names in $\Nc$ are concept descriptions;
	\item if $C$ and $D$ are concept descriptions, $C \sqcap D$ is a concept description;
	\item if $C$ is a concept description and $r \in \Nr$, $\exists r.C$ is a concept description;
	\item if $a \in \Ni$, $\{a\}$ is a concept description.
\end{itemize}

To represent knowledge using concept descriptions, we need to define facts (axioms and role inclusions) and assertions.

An \emph{axiom}, or a \emph{general concept inclusion} (GCI), is an expression of the form $C \isa D$, where $C$ and $D$ are concept inclusions. Also, we write $C \equiv D$ to represent the axioms $C \isa D$ and $D \isa C$. A \emph{role inclusion} (RI) is an expression of the form $r_1 \circ \cdots \circ r_k \isa r$, where $r_1, \dots r_k, r \in \Nr$.  The symbol ``$\circ$'' denotes composition of binary relations. A \emph{constraint box} (CBox) is a finite set of GCIs and a finite set of RIs.

Similarly, a \emph{concept assertion} is an expression of the form $C(a)$ and a \emph{role assertion}, $r(a, b)$, where $C$ is a concept description, $a, b \in \Ni$ and $r \in \Nr$. A finite set of concept assertions and role assertions is an \emph{assertional box} (ABox).

Then, an $\elpp$ \emph{knowledge base} $\K$ (KB) is a pair ($\mathcal{C}, \mathcal{A}$), where $\mathcal{C}$ is a CBox and $\mathcal{A}$ is an~ABox.

\subsection{Semantics}
The semantics of $\elpp$ are given by \emph{interpretations} $\I = (\Delta^\I, \cdot^\I)$. The \emph{domain} $\Delta^\I$ is a non-empty set of individuals, and the \emph{interpretation function} $\cdot^\I$ maps each concept name $A \in \Nc$ to a subset $A^\I$ of $\Delta^\I$, each role name $r$ to a binary relation $r^\I \subseteq \Delta^\I \times \Delta^\I$, and each individual name $a$ to an element $a^\I \in \Delta^\I$. The extension of $\cdot^\I$ for an arbitrary concept description is inductively defined by the third column of \autoref{table:elpp-syntax-semant}.

\begin{table}
	\centering
	\begin{tabular}{@{}rcc@{}}
		\toprule
		Name                    & Syntax                              & Semantics                                                                                        \\
		\midrule
		top                     & $\top$                              & $\Delta^\I$                                                                                      \\
		bottom                  & $\bot$                              & $\emptyset$                                                                                      \\
		nominal                 & $\{a\}$                             & $\{a^\I\}$                                                                                       \\
		conjunction             & $C \sqcap D$                        & $C^\I \cap D^\I$                                                                                 \\
		existential restriction & $\exists r.C$                       & $\{ x \in \Delta^\I \, | \, \exists y \in \Delta^\I \, : \, (x, y) \in r^\I \land y \in C^\I \}$ \\
		GCI                     & $C \isa D$                          & $C^\I \subseteq D^\I$                                                                            \\
		RI                      & $r_1 \circ \cdots \circ r_k \isa r$ & $r_1^\I \circ \cdots \circ r_k^\I \subseteq r^\I$                                                \\
		concept assertion       & $C(a)$                              & $a^\I \in C^\I$                                                                                  \\
		role assertion          & $r(a, b)$                           & $(a^\I, b^\I) \in r^\I$                                                                          \\
		\bottomrule
	\end{tabular}
	\caption{Syntax and semantics of $\elpp$ without concrete domains}
	\label{table:elpp-syntax-semant}
\end{table}

The interpretation $\I$ \emph{satisfies}:
\begin{itemize}
	\item an axiom $C \isa D$ if $C^\I \subseteq D^\I$ (represented as $I \models C \isa D$);

	\item a RI $r_1 \circ \cdots \circ r_k \isa r$ if $r_1^\I \circ \cdots \circ r_k^\I \subseteq r^\I$ (represented as $I \models r_1 \circ \cdots \circ r_k \isa r$);

	\item an assertion $C(a)$ if $a^\I \in C^\I$ (represented as $I \models C(a)$);

	\item an assertion $r(a, b)$ if $(a^\I, b^\I) \in r^\I$ (represented as $I \models r(a, b)$).
\end{itemize}

Also, we say that $\I$ is a \emph{model} of:
\begin{itemize}
	\item a CBox $\cbox$ if it satisfies every axiom and RI in $\cbox$ (represented as $\I \models \cbox$);
	\item an ABox $\abox$ if it satisfies every assertion in $\abox$ (represented as $\I \models \abox$);
\end{itemize}

Then, an important problem in $\elpp$ is to determine its $consistency$, that is if $\abox$ and $\cbox$ have a common model, which is in PTime \citep{Baader2005a}.

\subsection{Normal form}
We can convert an $\elpp$ knowledge base into a normal form, in polynomial time, by introducing new concept and role names \citep{Baader2005a}.

First, there is no need of explicit ABox, because $\I \models C(a)\iff \I \models \{a\} \isa C$ and $\I \models r(a, b) \iff \{a\} \isa \exists r.\{b\}$. In other words, a knowledge base can be represented by just a CBox, by transforming assertions in axioms.

In addition, given a CBox $\cbox$, consider the set $\bc$ of \emph{basic concept descriptions}, which is the smallest set of concept descriptions that contains the top concept $\top$, all concept names used in $\cbox$ and all concepts of the form $\{a\}$ used in $\cbox$.

Then, every axiom can be represented in the following normal form, where $C_1, C_2 \in \bc$, $D \in \bc \cup \{\bot\}$:
\begin{align*}
	C_1            & \isa D             \tag{simple}           \\
	C_1            & \isa \exists r.C_2 \tag{existential-head} \\
	\exists r.C_1  & \isa D             \tag{existential-body} \\
	C_1 \sqcap C_2 & \isa D  \tag{conjunctive-body}
\end{align*}
And every RI are of the form $r \isa s$ or $r_1 \circ r_2 \isa s$.

\begin{example}
	\label{exmp:el-cbox-def}
	Consider the following CBox $\CC_{exa}$ representing the situation in \autoref{exmp:real-example}. On the left, we have basic knowledge of diseases and, on the right, the specific knowledge about Mary. Note that, for simplicity, it is not in normal form.

		{
			\sffamily
			\begin{center}
				\begin{minipage}{0,4\textwidth}
					\fontsize{10}{14}
					\selectfont
					Fever $\isa$ Symptom\\
					Cough $\isa$ Symptom\\
					COVID-19 $\isa$ Disease\\
					Symptom $\isa \, \exists$hasCause.Disease\\
					Patient $\isa\, \exists$hasSymptom.Symptom\\
					hasSymptom $\circ$ hasCause $\isa$ suspectOf\\
				\end{minipage}
				\hspace{10pt}
				\begin{minipage}{0,4\textwidth}
					\fontsize{10}{14}
					\selectfont
					$\{$mary$\}$ $\isa$ Patient\\
					$\{$s1$\}$ $\isa$ Fever $\sqcap$ Cough\\
					$\{$mary$\}$ $\isa \, \exists$hasSymptom.$\{$s1$\}$\\
				\end{minipage}
			\end{center}
		}

	Because CBoxes can only represent facts, there is no way to describe uncertain knowledge. Even though, in cases when which of them are true, we could define three axioms

	\begin{description}
		\item { \sffamily $Ax_1 := $ Fever $\isa \exists$hasCause.COVID-19}, when fever is actually caused by COVID-19;
		\item { \sffamily $Ax_2 :=$ Fever $\sqcap$ Cough $\isa \exists$hasCause.COVID-19 }, when both fever and cough are caused by COVID-19;
		\item { \sffamily $Ax_3 :=$ COVID-19 $\isa \bot$}, when there are no presence of COVID-19 in the hospital.
	\end{description}

	In the following sections, it will be presented how to add these axioms in a KB with probabilistic properties.
\end{example}

We want to model uncertain information using DLs. However, it has been proved that, by adding probabilistic reasoning capabilities to $\elpp$, the complexity reaches NP-completeness \citep{Fin2019b}. Then, it is necessary to reduce the expressiveness of this language.

\section{Graphic \texorpdfstring{$\el$}{����} (\texorpdfstring{$\gel$}{������})}
\label{sec:gel}

\emph{Graphic} $\el$ ($\gel$) is a fragment of $\elpp$ in which every axiom and RI are in normal form and only simple and existential-head axioms are allowed \citep{Fin2020}. The semantics are the same as that of $\elpp$.

\begin{example}
	\label{exmp:gel-def}
	Since there are conjunctive-body axioms in the CBox in \autoref{exmp:el-cbox-def}, we need to modify this knowledge in order to represent it in $\gel$. First, we substitute every concept description {\sffamily Fever $\sqcap$ Cough} by a new basic concept  {\sffamily FeverAndCough}. After that, we add axioms {\sffamily FeverAndCough $\isa$ Fever} and {\sffamily FeverAndCough $\isa$ Cough} to CBox.
\end{example}

The name of this fragment comes from the fact that each GCI can be represented as arrows in a graph where nodes are basic concepts. This representation is useful for the development of algorithms and it is used to define its SAT decision.

\subsection{Graphical representation}
\label{subsec:graph-repr}

Consider a $\gel$-CBox $\CC$ with $n_R$ roles, its graphical representation is a edge-labeled graph $G(\CC) := (N, E, \ell)$, where $N$ is a set of nodes, $E \subseteq N^2$ is a set of directed edges and $\ell: E \to \{0, 1, \dots, n_R\}$ is a labeling function.

In addition, \citet{Fin2020} defines the following notation. The set $E_i \subseteq E$ is the set of all edges $e$ such that $\ell(e) = i$. We write $X_1 \to_i X_2$ if there is an edge $(X_1, X_2) \in E_i$, and $X_1 \nto_i X_2$ if $(X_1, X_2) \not\in E_i$. The expression $X \to_i^* Y$ represents the reflexive transitive closure of $\to_i$, which is the existence of a path in the graph of size $\geq 0$, starting in $X$, ending in $Y$, and only going through edges in $E_i$, for $0 \leq i \leq n_R$. Finally, $X \rightsquigarrow Y$ represents a path from $X$ to $Y$ using any type of edge.     

Then, the graph $G(\cbox) = (N, E, \ell)$ can be constructed from a CBox $\cbox$ with the following steps:
\begin{enumerate}
	\item for each concept $\exists r.C$ in a existential-body axiom of the form $\exists r.C \isa D$ create a \emph{virtual concept} ``$\exists r.C$''. The set of all virtual concepts in $\CC$ is called $\textsf{VC}_\CC$;
	\item the set $N$ of nodes is obtained from the basic concepts of $\CC$, an initial-node symbol $Init$, the bottom concept $\bot$ and the set of virtual concepts of $\CC$ as follows:
	\[ N := \{Init, \bot\} \cup \textsf{BC}_\CC \cup \textsf{VC}_\CC; \]
	\item if $C \isa D \in \CC$ then $C \to_0 D$;
	\item if $C \isa \exists r_i.D$ then $C \to_i D$;
	\item $Init \to_0 \top$;\footnote{Talk about sat} 
	\item for every node of the form $\{a_i\} \in N$, $Init \to_0 \{a_i\}$.
\end{enumerate}

\begin{example}
	Consider the CBox in \autoref{exmp:gel-def} and the uncertain information in \autoref{exmp:el-cbox-def}. Its graphical representation is displayed in the \autoref{fig:gel-graph}. The $\to_0$-edges are represented by continuous black arrows, the $\to_i$-edges, $i \geq 1$, are represented by blue labeled arrows and red dotted arrows indicate that source is uncertain information.
\end{example}

\tikzstyle{block} = [draw, rectangle, rounded corners=.2cm, fill=black!5, draw=white]
\tikzstyle{dashblock} = [draw, rectangle, dashed, rounded corners=.2cm, fill=black!5, draw=black!30, line width=0.8pt]
\tikzstyle{arrow} = [>=latex, line width=0.8pt]

\begin{figure}[ht]
	\centering
	\small
	\begin{tikzpicture}[auto, node distance=1.5cm, ->]
		\tikzset{every node/.style={minimum height=0.75cm}}

		\sffamily
		\node (init) {Init};
		\node [below=of init] (s1) {$\{$s1$\}$};
		\node [below=of s1] (mary) {$\{$mary$\}$};

		\node [block, right=0.5cm of mary] (patient) {Patient};
		\node [block, right=of s1] (fevcough) {FeverAndCough};
		\node [block, below=0.5cm of fevcough] (cough) {Cough};
		\node [block, right=of fevcough] (fever) {Fever};
		\node [block, below=of fever] (symptom) {Symptom};
		\node [block, above=of fever] (covid) {COVID-19};
		\node [block, right=of fever] (disease) {Disease};
		% \node [dashblock, above=of covid] (suspofcov) {``suspectOfCOVID-19''};
		% \node [block, left=0.5cm of suspofcov] (covidpat) {covidPatient};
		\node [above=of fevcough, yshift=-0.15cm] (bot) {$\bot$};

		\path [arrow]
		(init) edge (s1)
		(init) edge[bend right = 70] (mary)
		(s1) edge (fevcough)
		(mary) edge (patient)
		(covid) edge[bend right = -45] (disease)
		(fevcough) edge (fever)
		(fevcough) edge (cough)
		(cough) edge (symptom)
		(fever) edge (symptom)
		% (suspofcov) edge (covidpat)
		;

		\path [arrow, NavyBlue]
		(mary) edge node [midway, fill=white, anchor=center] {hasSymptom} (s1)
		(patient) edge node [midway, fill=white, anchor=center] {hasSymptom} (symptom)
		% (covidpat) edge[bend right = 40] node [near start, fill=white, anchor=center] {suspectOf} (covid.170)
		(symptom) edge[bend right = 45] node [midway, fill=white, anchor=center] {hasCause} (disease)
		% (suspofcov) edge[bend right = 60] node [midway, fill=white, anchor=center] {suspectOf} (covid)
		% (covid) edge[bend right = 60] node [midway, fill=white, anchor=center] {$u_{\text{suspectOf.Covid}}$} (suspofcov)
		;

		\path [arrow, dotted, FireBrick]
		(fevcough) edge node [midway, fill=white, anchor=center] {hasCause} (covid)
		(fever) edge node [midway, fill=white, anchor=center] {hasCause} (covid)
		;

		\path [arrow, dotted, FireBrick]
		(covid.190) edge (bot)
		;

	\end{tikzpicture}
	\caption{Graphical representation of the ontology in \autoref{exmp:gel-def}.}

	\label{fig:gel-graph}
\end{figure}

\subsection{SAT decision}

\improvement[inline]{Need to describe the graph completion rules}

\todo[inline]{Question about this part: do we really need the graph completion to define the SAT decision? Isn't it true that a $\gel$ CBox $\CC$ is \textbf{unsatisfiable}  iff $Init \rightsquigarrow \bot$  (using any type of edge) in a graph $G(\CC)$ \textbf{without} completion? I couldn't find any counterexample...}
After the graph completion, the satisfiability (SAT) decision of a $\gel$ CBox can be reduced to a path search in a graph, that is, a $\gel$ CBox is unsatisfiable iff $Init \to_0^* \bot$. 

\section{MaxSAT for \texorpdfstring{$\gel$}{������}}
\label{sec:maxgel}

Before focusing on the probabilistic extension for $\gel$, we need to define its MaxSAT problem, which will compose further the probabilistic reasoner.

The \emph{weighted partial maximum satisfiability problem for} $\gel$ ($\gel$-MaxSAT) can be defined as follows: given a potentially inconsistent weighted CBox $\cbox$, we want to find the maximal satisfiable subset of axioms; since it is partial, some axioms must be present in this subset. Usually, partiality can be modeled assigning infinite weights to the axioms that must not be excluded.

A \emph{weighted} CBox is a pair $\tuple{\cbox, w}$ where $\cbox$ is a CBox and $w: \cbox \to \Ratios \cup \{\infty\}$ is a weight function, which maps axioms in $\cbox$ to weights. The infinite weight is used to represent axioms that must not be excluded in the maximal satisfiable subset. Also, it is defined that RIs of the form $r \isa s$ and $r_1 \circ r_2 \isa s$ always have infinite weight.

Then, given a weighted CBox $\tuple{\cbox, w}$, a solution for the weighted partial $\gel$-MaxSAT problem is a set $\cbox_{max} \subseteq \cbox$ such that:
\begin{itemize}
	\item $\cbox_{max}$ is satisfiable; and
	\item $\cbox_{max} \models C \isa D$ if $w(C \isa D) = \infty$; and
	\item the sum of finite weights in $\cbox_{max}$ is maximal.
\end{itemize}

\section{Probabilistic \texorpdfstring{$\gel$}{������}}
\label{sec:pgel}

Probability in $\gel$ is constructed from a \emph{probability function} $P$ \citep{Fin2020}. Consider a finite number of interpretation, $\I_1, \dots, \I_m$, we define the probability function $P: \{\I_1, \dots, \I_m\} \to \Ratios$, such that $P(\I_i) \geq 0$ and $\sum_{i = 1}^m P(\I_i) = 1$. We can also define the probability of an axiom $C \isa D$ as follows

\[
	P(C \isa D) = \sum_{\I_i \models C \isa D} P(\I_i).
\]

A \emph{probabilistic knowledge base} is a pair $\tuple{\CC, \PP}$, where $\CC$ is a CBox and $\PP$ is a PBox. A PBox is a set of $k$ linear constraints over $n$ axioms, of the form

\begin{equation}
	\label{eq:pbox-restric}
	\sum_{j = 1}^n a_{ij} \cdot P(C_j \isa D_j) \leq b_i; \quad 1 \leq i \leq k.
\end{equation}

We can define the \emph{satisfiability problem} for this probabilistic KB (PGEL-SAT) as deciding if it is consistent or not. If it is consistent, the solution is a set of interpretations $\{\I_1, \dots, \I_m \}$ and a probability function $P: \{\I_1, \dots, \I_m \} \to \Ratios^+$ such that $\sum_{i = 1}^m P(\I_i) = 1$, $P(C \isa D) = 1$ for $C \isa D \in \CC$ (axioms in CBox are certain) and P verifies all linear constraints in $\PP$.
% 
\begin{example}
	Now we can model the uncertain situation stated in \autoref{exmp:real-example} using the probability knowledge base $\tuple{\CC_{exa}, \PP_{exa}}$, where $\CC_{exa}$ is the CBox from \autoref{exmp:gel-def} and $\PP_{exa}$ is given by
	\begin{align*}
		\PP_{exa} := \{ \quad P(Ax_2) - P(Ax_1) & = 0.2,           \\
		P(Ax_3)                                 & = 0.9  \quad \}.
	\end{align*}

	Then, we need a polynomial algorithm to find if this probabilistic KB is consistent.
\end{example}
% 
\subsection{Linear algebraic view}
\label{subsec:lin-alg-view}

The PGEL-SAT problem was also defined by \citet{Fin2020} in a linear algebraic view, which is useful to develop its polynomial reasoning algorithm. It was shown that a probabilistic KB $\tuple{\CC, \PP}$ is satisfiable iff the linear equation $C \cdot x = d$ has a solution $x \geq 0$, where
% 
\begin{equation}
	\label{eq:pbox-linear}
	C := \begin{bmatrix}
		-I_n           & M_{n \times m} \\
		A_{k \times n} & 0_{k \times m} \\
		0'_n           & 1'_{m}
	\end{bmatrix}
	%
	\quad
	x := \begin{bmatrix}
		p_n \\
		\pi_m
	\end{bmatrix}
	%
	\quad
	d := \begin{bmatrix}
		0_n \\
		b_k \\
		1
	\end{bmatrix}
\end{equation}

and
\begin{itemize}
	\item $A_{k \times n}$ is a $k \times n$ matrix whose elements $a_{ij}$ are given by \autoref{eq:pbox-restric};
	\item $b_k$ is a $k$ vector whose elements $b_i$ are also given by \autoref{eq:pbox-restric};
	\item $M_{n \times m}$ is a $n \times m$ matrix given by the following steps:

	      Consider an interpretation $\I$ model of $\CC$, also called $C$-\emph{satisfiable interpretation}, its corresponding vector in $\PP$ is  a $\{0, 1\}$-vector $y$ such that $y_i = 1$ iff $\I \models C_i \isa D_i$, for $1 \leq i \leq n$.

	      Then, given a set of interpretations $\I_1, \dots, \I_m$, we define $M_{n \times m}$ a matrix whose column $M^j$ is $\I_j$'s corresponding vector in $\PP$;
	\item $I_n$ is the $n$-dimensional identity matrix;
	\item $0_n$ is a column $0$-vector of size $n$ (similarly for $1_n$);
	\item $0'_n$ is the previous vector's transpose;
	\item $0_{k \times m}$ is a $0$-matrix of shape $k \times m$;
	\item $p_n$ is a vector of size $n$ which corresponds to the probability of axioms occurring in \autoref{eq:pbox-restric};
	\item $\pi_m$ is a vector of size $m$ which corresponds to the probability distribution over interpretations $\I_1, \dots, \I_m$.
\end{itemize}

\todo[inline]{Intuition about this matrices}

Thus, we can use techniques for solving linear equations to find a tractable algorithm for PGEL-SAT.
  %!TeX root=../tese.tex

\chapter{Development}
\label{cap:development}

\section{OWL parser}

\section{Knowledge Base}

\section{GEL-MaxSAT}

\section{Linear solver}

\section{PGEL-SAT reasoner}

  %!TeX root=../tese.tex

\chapter{Experiments}
\label{cap:experiments}

  %!TeX root=../tese.tex

\chapter{Results}
\label{cap:results}

  %!TeX root=../tese.tex

\chapter{Related work}
\label{cap:relatedwork}

The problem of probabilistic reasoning and extensions in logics to deal with uncertainty have been studied for several decades. The first known proposal of PSAT, for propositional formulas, is attributed to \citet{boole1854investigation} and it has already been shown to be NP-Complete \citep{georgakopoulos1988probabilistic}.

In the relational domain, the literature contain several logics with probabilistic reasoning capabilities although they have led to intractable decision problems.  Some of them extend the already intractable $\mathcal{ALC}$, with probabilistic constrains over concepts \citep{heinsohn1994probabilistic, lukasiewicz2008expressive, GutierrezBasultoEA11}. For the expressive and lightweight $\el$-family, some extensions such as \citet{gutierrez2017probabilistic,ceylan2017bayesian} have led to \textsc{ExpTime}-hard or PP-complete probabilistic reasoning; futhermore, NP-completeness can be achieved with probability capabilities over axioms \citep{Fin2019b}.

On the other hand, many results implies that the research on Max-SAT has a impact on the solutions of PSAT problems \citep{andersen2001easy}. Also, there was already proposed a MaxSAT-solver for a propositional fragment of horn logic by a max-flow/min-cut formulation \citep{jaumard1987complexity}. Thus, it is expected to ask if one could also take such results to a relational domain.

  %!TeX root=../tese.tex

\chapter{Conclusion}
\label{cap:conclusion}


  \par

  %%%%%%%%%%%%%%%%%%%%%%%%%%%% APÊNDICES E ANEXOS %%%%%%%%%%%%%%%%%%%%%%%%%%%%%%%%

  % Um apêndice é algum conteúdo adicional de sua autoria que colabora com a
  % ideia geral do texto mas que, por alguma razão, não precisa fazer parte
  % da sequência do discurso; por exemplo, a demonstração de um teorema, as
  % perguntas usadas em uma pesquisa qualitativa etc.
  %
  % Um anexo é um documento que não é de sua autoria mas que é relevante para
  % a tese; por exemplo, a especificação do padrão que o trabalho discute.
  %
  % Os comandos appendix e annex reiniciam a numeração de capítulos e passam
  % a numerá-los com letras. "annex" não faz parte de nenhuma classe padrão,
  % ele foi criado para este modelo (em annex.sty e utils.tex). Se o
  % trabalho não tiver apêndices ou anexos, remova estas linhas.
  %
  % Diferentemente de \mainmatter, \backmatter etc., \appendix e \annex não
  % forçam o início de uma nova página. Em geral isso não é importante, pois
  % o comando seguinte costuma ser "\chapter", mas pode causar problemas com
  % a formatação dos cabeçalhos. Assim, vamos forçar uma nova página antes
  % de cada um deles.

  %%%% Apêndices %%%%
  % \makeatletter
  % \if@openright\cleardoublepage\else\clearpage\fi
  % \makeatother

  % Este formato está definido na package imeusp-headers.
  % \pagestyle{appendix}

  % \appendix

  %\input{conteudo-exemplo/apendices}
  % \input{conteudo/apendices}
  % \par


  %%%%%%%%%%%%%%%%%%%%%%%%%%%%%% SEÇÕES FINAIS %%%%%%%%%%%%%%%%%%%%%%%%%%%%%%%%%%%

  % Aqui vão a bibliografia, índice remissivo e outras seções similares.

  % O comando backmatter desabilita a numeração de capítulos.
  \backmatter

  % Este formato está definido na package imeusp-headers
  \pagestyle{backmatter}

  % Espaço adicional no sumário antes das referências / índice remissivo
  \addtocontents{toc}{\vspace{2\baselineskip plus .5\baselineskip minus .5\baselineskip}}

  % A bibliografia é obrigatória
  %%%%%%%% Bibliografia com biblatex (preferido): %%%%%%%%

  \printbibliography[
    title=\refname\label{bibliography}, % "Referências", recomendado pela ABNT
    heading=bibintoc,
  ]

  % imprime o índice remissivo no documento (opcional)
  % \printindex

\end{document}
